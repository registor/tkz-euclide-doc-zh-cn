% tkz-obj-eu-sectors.tex
% Copyright 2020  Alain Matthes
% This work may be distributed and/or modified under the
% conditions of the LaTeX Project Public License, either version 1.3
% of this license or (at your option) any later version.
% The latest version of this license is in
%   http://www.latex-project.org/lppl.txt
% and version 1.3 or later is part of all distributions of LaTeX
% version 2005/12/01 or later.
%
% This work has the LPPL maintenance status “maintained”.
%
% The Current Maintainer of this work is Alain Matthes.
%  utf8 encoding
\def\fileversion{3.06c}
\def\filedate{2020/04/06}
\typeout{2020/04/06 3.06c tkz-obj-eu-sectors.tex}
\makeatletter
%<-----------------------    Sectors         ------------------------------–>
\tikzset{sstyle/.style={#1}}
\gdef\tkz@nums{0}
\pgfkeys{/drawsector/.cd,
	towards/.code               = {\def\tkz@nums{0}},
	rotate/.code                = {\def\tkz@nums{1}},
	R/.code                     = {\def\tkz@nums{2}},
    R with nodes/.code          = {\def\tkz@nums{3}},
	towards,
  /drawsector/.search also  = {/tikz},
}

\def\tkzDrawSector{\pgfutil@ifnextchar[{\tkz@DrawSector}{\tkz@DrawSector[]}}
\def\tkz@DrawSector[#1](#2,#3)(#4){%
\begingroup
\pgfqkeys{/drawsector}{#1}
\ifcase\tkz@nums%
   \tkzDrawSectorN[#1](#2,#3)(#4)
\or% 1
   \tkzDrawSectorRotate[#1](#2,#3)(#4)
\or% 2
   \tkzDrawSectorRAngles[#1](#2,#3)(#4)
\or% 3
   \tkzDrawSectorR[#1](#2,#3)(#4)
\fi
\endgroup
}
\def\tkzDrawSectorRAngles{\pgfutil@ifnextchar[{\tkz@DrawSectorRAngles}{%
                                               \tkz@DrawSectorRAngles[]}}
\def\tkz@DrawSectorRAngles[#1](#2,#3)(#4,#5){%
\begingroup
  \pgfmathparse{#4}\edef\tkz@FirstAngle{\pgfmathresult}%
  \pgfmathparse{#5}\edef\tkz@SecondAngle{\pgfmathresult}%
  \pgfmathgreaterthan{\tkz@FirstAngle}{0}
  \ifdim\pgfmathresult pt=1 pt\relax%
    \pgfmathgreaterthan{\tkz@FirstAngle}{\tkz@SecondAngle}
    \ifdim\pgfmathresult pt=1 pt\relax%
     \pgfmathsubtract{\tkz@FirstAngle}{360}
     \edef\tkz@FirstAngle{\pgfmathresult}%
 \fi
 \else
     \pgfmathgreaterthan{\tkz@FirstAngle}{\tkz@SecondAngle}
    \ifdim\pgfmathresult pt=1 pt\relax%
     \pgfmathadd{\tkz@SecondAngle}{360}
     \edef\tkz@SecondAngle{\pgfmathresult}%
 \fi
 \fi
  \draw[shift = {(#2)},/drawsector/.cd,#1]%
    (#2) --(\tkz@FirstAngle:#3) arc (\tkz@FirstAngle:\tkz@SecondAngle:#3)--cycle;
 \endgroup
  } %  revoir il y a plus simple que ce shift !!!
%<--------------------------------------------------------------------------–>
% \tkzDrawSectorN(O,A)(B)
%<--------------------------------------------------------------------------–>
\def\tkzDrawSectorN{\pgfutil@ifnextchar[{\tkz@DrawSectorN}{%
                                         \tkz@DrawSectorN[]}}
\def\tkz@DrawSectorN[#1](#2,#3)(#4){%
\begingroup
 \tkzCalcLength(#2,#3)\tkzGetLength{tkz@radius}
 \tkzFindSlopeAngle(#2,#3)\tkzGetAngle{tkz@FirstAngle}
 \tkzFindSlopeAngle(#2,#4)\tkzGetAngle{tkz@SecondAngle}
 \tkz@DrawSectorRAngles[#1](#2,\tkz@radius pt)(\tkz@FirstAngle,%
                                               \tkz@SecondAngle)
\endgroup
}
%<--------------------------------------------------------------------------–>
% \tkzDrawSectorRotate(O,A)(angle de rotation) ??? pb avec angle negatif
%<--------------------------------------------------------------------------–>
\def\tkzDrawSectorRotate{\pgfutil@ifnextchar[{\tkz@DrawSectorRotate}{%
                                              \tkz@DrawSectorRotate[]}}
\def\tkz@DrawSectorRotate[#1](#2,#3)(#4){%
\begingroup
    \tkzCalcLength(#2,#3)\tkzGetLength{tkz@radius}
    \tkzFindSlopeAngle(#2,#3)\tkzGetAngle{tkz@FirstA}
    \pgfmathadd{\tkz@FirstA}{#4}
      \edef\tkz@SecondA{\pgfmathresult}
    \pgfmathgreaterthan{#4}{0}
  \ifdim\pgfmathresult pt=1 pt\relax%
   \tkz@DrawSectorRAngles[#1](#2,\tkz@radius pt)(\tkz@FirstA,\tkz@SecondA)
  \else
  \tkz@DrawSectorRAngles[#1](#2,\tkz@radius pt)(\tkz@SecondA,\tkz@FirstA)
  \fi
\endgroup
}
%<--------------------------------------------------------------------------–>
%  \tkzDrawSectorAngles(O,A)(ang1,ang2)
%<--------------------------------------------------------------------------–>
\def\tkzDrawSectorAngles{\pgfutil@ifnextchar[{\tkz@DrawSectorAngles}{%
                                              \tkz@DrawSectorAngles[]}}
\def\tkz@DrawSectorAngles[#1](#2,#3)(#4,#5){%
\begingroup
    \tkzCalcLength(#2,#3)\tkzGetLength{tkz@radius}
    \tkz@DrawSectorRAngles[#1](#2,\tkz@radius pt)(#4,#5)
\endgroup
}
%<--------------------------------------------------------------------------–>
%    \tkzDrawSectorRwithNodesAngles(O,A,B)(ang1,ang2)
%<--------------------------------------------------------------------------–>
\def\tkzDrawSectorRwithNodesAngles{\pgfutil@ifnextchar[{%
            					   \tkz@DrawSectorRwithNodesAngles}{%
            					   \tkz@DrawSectorRwithNodesAngles[]}}
\def\tkz@DrawSectorRwithNodesAngles[#1](#2,#3,#4)(#5,#6){%
\begingroup
    \tkzCalcLength(#3,#4)\tkzGetLength{tkz@radius}
    \tkz@DrawSectorRAngles[#1](#2,\tkz@radius pt)(#5,#6)
\endgroup
}
% %<--------------------------------------------------------------------------–>
% % \tkzDrawSectorRwithNodes(O,A,B)(C,D)  center O radius  AB from C to D
% %<--------------------------------------------------------------------------–>
% \def\tkzDrawSectorRwithNodes{\pgfutil@ifnextchar[{%
%             \tkz@DrawSectorRwithNodes}{%
%             \tkz@DrawSectorRwithNodes[]}}
% \def\tkz@DrawSectorRwithNodes[#1](#2,#3,#4)(#5,#6){%
% \begingroup
%     \tkzCalcLength(#3,#4){tkz@rad}
%     \tkzFindSlopeAngle(#2,#5)\tkzGetAngle{tkz@FirstAngle}
%     \tkzFindSlopeAngle(#2,#6)\tkzGetAngle{tkz@SecondAngle}
%     \tkz@DrawSectorRAngles[#1](#2,\tkz@rad)(\tkz@FirstAngle,\tkz@SecondAngle)
% \endgroup
% }
%<--------------------------------------------------------------------------–>
%  \tkzDrawSectorR(O, rad)(A,B)
%<--------------------------------------------------------------------------–>
\def\tkzDrawSectorR{\pgfutil@ifnextchar[{\tkz@DrawSectorR}{%
                                         \tkz@DrawSectorR[]}}
\def\tkz@DrawSectorR[#1](#2,#3)(#4,#5){%
\begingroup
  \tkzFindSlopeAngle(#2,#4)\tkzGetAngle{tkz@FirstAngle}
  \tkzFindSlopeAngle(#2,#5)\tkzGetAngle{tkz@SecondAngle}
  \tkz@DrawSectorRAngles[#1](#2,#3)(\tkz@FirstAngle,\tkz@SecondAngle)
\endgroup
}
%<--------------------------------------------------------------------------–>
%<--------------------------------------------------------------------------–>
%                                 Fill
%<--------------------------------------------------------------------------–>
%<--------------------------------------------------------------------------–>
\gdef\tkz@nums{0}
\pgfkeys{/@fillsector/.cd,
			towards/.code               = {\def\tkz@nums{0}},
			rotate/.code                = {\def\tkz@nums{1}},
			R/.code                     = {\def\tkz@nums{2}},
			R with nodes/.code          = {\def\tkz@nums{3}},
			towards,
			/@fillsector/.search also={/tikz},
}

\def\tkzFillSector{\pgfutil@ifnextchar[{\tkz@FillSector}{\tkz@FillSector[]}}
\def\tkz@FillSector[#1](#2,#3)(#4){%
\begingroup
\pgfqkeys{/@fillsector}{#1}
\ifcase\tkz@nums%
   \tkzFillSectorN[#1](#2,#3)(#4)
\or% 1
   \tkzFillSectorRotate[#1](#2,#3)(#4)
\or% 2
   \tkzFillSectorRAngles[#1](#2,#3)(#4)
\or% 3
   \tkzFillSectorR[#1](#2,#3)(#4)
\fi
\endgroup
}


\def\tkzFillSectorRAngles{\pgfutil@ifnextchar[{\tkz@FillSectorRAngles}{%
                                               \tkz@FillSectorRAngles[]}}
\def\tkz@FillSectorRAngles[#1](#2,#3)(#4,#5){%
\begingroup
  \pgfmathparse{#4}\edef\tkz@FirstAngle{\pgfmathresult}%
  \pgfmathparse{#5}\edef\tkz@SecondAngle{\pgfmathresult}%
   \pgfmathgreaterthan{\tkz@FirstAngle}{0}
   \ifdim\pgfmathresult pt=1 pt\relax%
     \pgfmathgreaterthan{\tkz@FirstAngle}{\tkz@SecondAngle}
     \ifdim\pgfmathresult pt=1 pt\relax%
      \pgfmathsubtract{\tkz@FirstAngle}{360}
      \edef\tkz@FirstAngle{\pgfmathresult}%
  \fi
  \else
      \pgfmathgreaterthan{\tkz@FirstAngle}{\tkz@SecondAngle}
     \ifdim\pgfmathresult pt=1 pt\relax%
      \pgfmathadd{\tkz@SecondAngle}{360}
      \edef\tkz@SecondAngle{\pgfmathresult}%
  \fi\fi
  \pgfmathsubtract{\tkz@FirstAngle}{\tkz@delta}
  \edef\tkz@FirstAngle{\pgfmathresult}%
  \pgfmathadd{\tkz@SecondAngle}{\tkz@delta}
  \edef\tkz@SecondAngle{\pgfmathresult}
  \path[shift  = {(#2)},/@fillsector/.cd,fill=\tkz@fillcolor,#1] (#2) -- %
      (\tkz@FirstAngle:#3) arc (\tkz@FirstAngle:\tkz@SecondAngle:#3)--cycle;
\endgroup
}
%<--------------------------------------------------------------------------–>
% \tkzFillSectorN(O,A)(B)
%<--------------------------------------------------------------------------–>
\def\tkzFillSectorN{\pgfutil@ifnextchar[{\tkz@FillSectorN}{\tkz@FillSectorN[]}}
\def\tkz@FillSectorN[#1](#2,#3)(#4){%
\begingroup
  \tkzCalcLength(#2,#3)\tkzGetLength{tkz@radius}
  \tkzFindSlopeAngle(#2,#3)\tkzGetAngle{tkz@FirstAngle}
  \tkzFindSlopeAngle(#2,#4)\tkzGetAngle{tkz@SecondAngle}
  \tkz@FillSectorRAngles[#1](#2,\tkz@radius pt)(\tkz@FirstAngle,%
                                                \tkz@SecondAngle)
\endgroup
}
%<--------------------------------------------------------------------------–>
% \tkzFillSectorRotate(O,A)(angle de rotation) ??? pb avec angle negatif
%<--------------------------------------------------------------------------–>
\def\tkzFillSectorRotate{\pgfutil@ifnextchar[{\tkz@FillSectorRotate}{%
                                              \tkz@FillSectorRotate[]}}
\def\tkz@FillSectorRotate[#1](#2,#3)(#4){%
\begingroup
    \tkzCalcLength(#2,#3)\tkzGetLength{tkz@radius}
    \tkzFindSlopeAngle(#2,#3)\tkzGetAngle{tkz@FirstA}
    \pgfmathadd{\tkz@FirstA}{#4}
    \edef\tkz@SecondA{\pgfmathresult}
    \pgfmathgreaterthan{#4}{0}
  \ifdim\pgfmathresult pt=1 pt\relax%
   \tkz@FillSectorRAngles[#1](#2,\tkz@radius pt)(\tkz@FirstA,\tkz@SecondA)
  \else
  \tkz@FillSectorRAngles[#1](#2,\tkz@radius pt)(\tkz@SecondA,\tkz@FirstA)
  \fi
   \endgroup
}
%<--------------------------------------------------------------------------–>
%  \tkzFillSectorAngles(O,A)(ang1,ang2)
%<--------------------------------------------------------------------------–>
\def\tkzFillSectorAngles{\pgfutil@ifnextchar[{\tkz@FillSectorAngles}{%
                                              \tkz@FillSectorAngles[]}}
\def\tkz@FillSectorAngles[#1](#2,#3)(#4,#5){%
\begingroup
    \tkzCalcLength(#2,#3)\tkzGetLength{tkz@radius}
    \tkz@FillSectorRAngles[#1](#2,\tkz@radius pt)(#4,#5)
\endgroup
}
%<--------------------------------------------------------------------------–>
%  \tkzFillSectorR(O, rad)(A,B)
%<--------------------------------------------------------------------------–>
\def\tkzFillSectorR{\pgfutil@ifnextchar[{\tkz@FillSectorR}{%
                                         \tkz@FillSectorR[]}}
\def\tkz@FillSectorR[#1](#2,#3)(#4,#5){%
\begingroup
  \tkzFindSlopeAngle(#2,#4)\tkzGetAngle{tkz@FirstAngle}
  \tkzFindSlopeAngle(#2,#5)\tkzGetAngle{tkz@SecondAngle}
  \tkz@FillSectorRAngles[#1](#2,#3)(\tkz@FirstAngle,\tkz@SecondAngle)
\endgroup
}
%<--------------------------- Clip Sector ----------------------------------–>
%
%<--------------------------------------------------------------------------–>
\def\tkz@ClipSectorRAngles(#1,#2)(#3,#4){%
\begingroup
  \pgfmathparse{#3}\edef\tkz@FirstAngle{\pgfmathresult}%
  \pgfmathparse{#4}\edef\tkz@SecondAngle{\pgfmathresult}%
  \pgfmathgreaterthan{\tkz@FirstAngle}{0}
  \ifdim\pgfmathresult pt=1 pt\relax%
    \pgfmathgreaterthan{\tkz@FirstAngle}{\tkz@SecondAngle}
    \ifdim\pgfmathresult pt=1 pt\relax%
     \pgfmathsubtract{\tkz@FirstAngle}{360}
     \edef\tkz@FirstAngle{\pgfmathresult}%
 \fi
 \else
     \pgfmathgreaterthan{\tkz@FirstAngle}{\tkz@SecondAngle}
    \ifdim\pgfmathresult pt=1 pt\relax%
     \pgfmathadd{\tkz@SecondAngle}{360}
     \edef\tkz@SecondAngle{\pgfmathresult}%
 \fi
 \fi
  \clip[shift = {(#1)}] (#1) --(\tkz@FirstAngle:#2)%
        arc(\tkz@FirstAngle:\tkz@SecondAngle:#2) -- cycle;
 \endgroup
  } %  revoir il y a plus simple que ce shift !!!
%<--------------------------------------------------------------------------–>
%
%<--------------------------------------------------------------------------–>
\def\tkz@ClipSectorRotate(#1,#2)(#3){%
\begingroup
    \tkzCalcLength(#1,#2)\tkzGetLength{tkz@radius}
    \tkzFindSlopeAngle(#1,#2)\tkzGetAngle{tkz@FirstA}
    \pgfmathadd{\tkz@FirstA}{#3}
    \edef\tkz@SecondA{\pgfmathresult}
    \pgfmathgreaterthan{#3}{0}
  \ifdim\pgfmathresult pt=1 pt\relax%
   \tkz@ClipSectorRAngles(#1,\tkz@radius pt)(\tkz@FirstA,\tkz@SecondA)
  \else
  \tkz@ClipSectorRAngles(#1,\tkz@radius pt)(\tkz@SecondA,\tkz@FirstA)
  \fi
   \endgroup
}
%<--------------------------------------------------------------------------–>
%
%<--------------------------------------------------------------------------–>

\def\tkz@ClipSectorN(#1,#2)(#3){%
\begingroup
  \tkzCalcLength(#1,#2)\tkzGetLength{tkz@radius}
  \tkzFindSlopeAngle(#1,#2)\tkzGetAngle{tkz@FirstAngle}
  \tkzFindSlopeAngle(#1,#3)\tkzGetAngle{tkz@SecondAngle}
  \tkz@ClipSectorRAngles(#1,\tkz@radius pt)(\tkz@FirstAngle,\tkz@SecondAngle)
\endgroup
}
%<--------------------------------------------------------------------------–>
%<--------------------------------------------------------------------------–>
\def\tkz@numcs{0}
\pgfkeys{/clipsector/.cd,
			towards/.code               = {\def\tkz@numcs{0}},
			rotate/.code                = {\def\tkz@numcs{1}},
			R/.code                     = {\def\tkz@numcs{2}},
}
%<--------------------------------------------------------------------------–>
\def\tkzClipSector{\pgfutil@ifnextchar[{\tkz@ClipSector}{%
                                        \tkz@ClipSector[]}}

\def\tkz@ClipSector[#1](#2,#3)(#4){%
\begingroup
\pgfkeys{/clipsector/.cd}
\pgfqkeys{/clipsector}{#1}
\ifcase\tkz@numcs%
   \tkz@ClipSectorN(#2,#3)(#4)
  \or% 1
   \tkz@ClipSectorRotate(#2,#3)(#4)
  \or% 2
   \tkz@ClipSectorRAngles(#2,#3)(#4)
  \fi
\endgroup
}
\makeatother
\endinput
