% tkz-obj-eu-polygons.tex
% Copyright 2020  Alain Matthes
% This work may be distributed and/or modified under the
% conditions of the LaTeX Project Public License, either version 1.3
% of this license or (at your option) any later version.
% The latest version of this license is in
%   http://www.latex-project.org/lppl.txt
% and version 1.3 or later is part of all distributions of LaTeX
% version 2005/12/01 or later.
%
% This work has the LPPL maintenance status “maintained”.
%
% The Current Maintainer of this work is Alain Matthes.
%  utf8 encoding
\def\fileversion{3.06c}
\def\filedate{2020/04/06}
\typeout{2020/04/06 3.06c tkz-obj-eu-polygons.tex}
% bug in regular polygon side 2020/03/18
\makeatletter
%<--------------------------------------------------------------------------–>
%                                 Polygon
%<--------------------------------------------------------------------------–>
%<---------------------------   square  ------------------------------------–>
%
%<--------------------------------------------------------------------------–>
\def\tkzDefSquare(#1,#2){
\begingroup
 	\tkzURotateAngle(#2,-90)(#1)
  \pgfnodealias{tkzFirstPointResult}{tkzPointResult}
 	\tkzURotateAngle(#1, 90)(#2)
  \pgfnodealias{tkzSecondPointResult}{tkzPointResult}
\endgroup
}
%<---------------------   parallélogramme  ---------------------------------–>
%
%<--------------------------------------------------------------------------–>
\def\tkzDefParallelogram(#1,#2,#3){
\begingroup
\tkzDefPointWith[colinear= at #3](#2,#1)
\endgroup
}

%<-------------------------- gold rectangle -------------------------------–>
%
%<--------------------------------------------------------------------------–>

\def\tkzDefGoldRectangle(#1,#2){
\begingroup
  \tkzDefPointWith[K=-\tkzInvPhi](#2,#1)
  \pgfnodealias{tkzFirstPointResult}{tkzPointResult}
  \tkzDefPointWith[K=\tkzInvPhi](#1,#2)
  \pgfnodealias{tkzSecondPointResult}{tkzPointResult}
\endgroup
}
\def\tkzDrawGoldRectangle{\pgfutil@ifnextchar[{\tkz@DrawGoldRectangle}{%
                                         		\tkz@DrawGoldRectangle[]}}

%<----------------------------   Regular Polygon   -------------------------–>
\def\tkz@numregpol{0}
\pgfkeys{/defregpoly/.cd,
			name/.store in    	=  \tkz@regpolname,
			sides/.store in   	=  \tkz@regpolsides,
			center/.code  		  =  \def\tkz@numregpol{0},
			side/.code    	   	=  \def\tkz@numregpol{1},
			name/.default				=  P,
			sides/.default			=  5,
      center
}
\def\tkzDefRegPolygon{\pgfutil@ifnextchar[{\tkz@DefRegPolygon}{\tkz@DefRegPolygon[]}}
\def\tkz@DefRegPolygon[#1](#2,#3){%
\begingroup
\pgfqkeys{/defregpoly}{#1}
\ifcase\tkz@numregpol%
  \tkzRegPolygonCenter(#2,#3)
  \or%
  \tkzRegPolygonSide(#2,#3)
 \fi
\endgroup
}
\def\tkzRegPolygonCenter(#1,#2){
\begingroup
   \foreach \v in {1,2,...,\tkz@regpolsides}
   {%
   \pgfmathsetmacro{\tkz@regangle}{360/\tkz@regpolsides*(\v-1)}
   \tkzDefPointBy[rotation= center #1 angle \tkz@regangle](#2)
   \pgfnodealias{\tkz@regpolname\v}{tkzPointResult}
   }
\endgroup
}
\def\tkzRegPolygonSide(#1,#2){
\begingroup
% get the center
\pgfmathsetmacro{\tkz@regangle}{360/\tkz@regpolsides*(\tkz@regpolsides-1)}
\pgfmathsetmacro{\tkz@regangleside}{(180-\tkz@regangle)/2}
\tkzDefMidPoint(#1,#2)
\pgfnodealias{tkz@tempPt}{tkzPointResult}
\tkzCalcLength[cm](tkz@tempPt,#1) \tkzGetLength{tkz@len}
\pgfmathsetmacro{\tkz@inscriberadius}{%
  \tkz@len*tan(90*(\tkz@regpolsides-2)/\tkz@regpolsides)}
\tkzDefPointWith[orthogonal normed,K=\tkz@inscriberadius](tkz@tempPt,#2)
\pgfnodealias{tkz@RegPolCenter}{tkzPointResult}
\tkzRegPolygonCenter(tkz@RegPolCenter,#1)
\pgfnodealias{tkzPointResult}{tkz@RegPolCenter}
\endgroup
}

%<----------------------------   CLIP       --------------------------------–>
%
%<--------------------------------------------------------------------------–>
\def\tkzClipPolygon(#1,#2){%
\path[clip] (#1)
   \foreach \pt in {#1,#2}{--(\pt)}--cycle;
}

 \def\tkzClipOutPolygon(#1,#2){\clip[tkzreverseclip] (#1)
 \foreach \pt in {#1,#2}{--(\pt)}--cycle;
}
\makeatother
\endinput
